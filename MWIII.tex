\documentclass{beamer}

\usepackage{xcolor,cancel}

\newcommand\hcancel[2][black]{\setbox0=\hbox{$#2$}%
\rlap{\raisebox{.45\ht0}{\textcolor{#1}{\rule{\wd0}{1pt}}}}#2} 


\usepackage{beamerthemesplit}

\usepackage{calligra}
\usepackage{comment}
\usepackage[normalem]{ulem}
%\usetheme{AnnArbor}
%\usetheme{default}
%\usetheme{Warsaw}
\usetheme{Boadilla} 
%\setbeamersize{text margin left=1cm,text margin right=1cm}
%\setbeamercolor{background canvas}{bg=lightgray}
%\usecolortheme[RGB={1,1,1}]{structure}

\makeatother
\setbeamertemplate{footline}
{
  \leavevmode%
  \hbox{%
  \begin{beamercolorbox}[foregroundcolor=white,wd=.90\paperwidth,ht=2.25ex,dp=1ex,center]{title in head/foot}%
    \usebeamerfont{author in head/foot}\insertshortauthor
  \end{beamercolorbox}%
  \begin{beamercolorbox}[wd=.10\paperwidth,ht=2.25ex,dp=1ex,right]{title in head/foot}%
%    \usebeamerfont{title in head/foot}\insertshorttitle\hspace*{3em}
    \insertframenumber{} / \inserttotalframenumber\hspace*{1ex}
  \end{beamercolorbox}}%
  \vskip0pt%
}
\makeatletter
\setbeamertemplate{navigation symbols}{}


%\usepackage{amssymb} 
%\usepackage{amsmath} 
%\usepackage{amscd}
%\usepackage{amsbsy}
%\usepackage{mathabx}
%\usepackage{MnSymbol}
\usepackage[all,cmtip]{xy}

%\usepackage[matrix,arrow]{xy}
%\usepackage[small,nohug,heads=vee]{diagrams}
%\diagramstyle[labelstyle=\scriptstyle]
\DeclareMathOperator{\Sym}{Sym}
\DeclareMathOperator{\Chab}{Chab}
\DeclareMathOperator{\disc}{disc}
\DeclareMathOperator{\genus}{genus}
\newcommand{\pdual}{\check{\mathbb{P}}^2}
\DeclareMathOperator{\Char}{char}
\DeclareMathOperator{\PGL}{PGL}
\DeclareMathOperator{\Aut}{Aut}
\DeclareMathOperator{\Cl}{Cl}
\DeclareMathOperator{\Disc}{Disc}
\DeclareMathOperator{\GL}{GL}
\DeclareMathOperator{\SL}{SL}
\DeclareMathOperator{\Sel}{Sel}
\DeclareMathOperator{\Ker}{Ker}
\DeclareMathOperator{\Img}{Im}
\DeclareMathOperator{\Res}{Res}
\DeclareMathOperator{\rank}{rank}
\DeclareMathOperator{\Pic}{Pic}
\DeclareMathOperator{\red}{red}
\newcommand{\vv}{{\upsilon}}
\DeclareMathOperator{\Div}{Div}
\DeclareMathOperator{\Princ}{Princ}
\DeclareMathOperator{\divisor}{div}
\DeclareMathOperator{\Alb}{Alb}
\DeclareMathOperator{\Gal}{Gal}
\DeclareMathOperator{\norm}{Norm}
\DeclareMathOperator{\trace}{Tr}
\DeclareMathOperator{\ord}{ord}
\DeclareMathOperator{\image}{im}
\DeclareMathOperator{\lcm}{lcm}
\DeclareMathOperator{\Reg}{Reg}
\newcommand{\crt}{\sqrt[3]{2}}
\newcommand{\bxi}{\boldsymbol{\xi}}
%\newcommand{\SHA}[1]{{\cyr X}(#1)}
%\newcommand{\p}{{\mathfrak P}}
%\newcommand{\A}{{\mathbb A}}
\newcommand{\mS}{\mathcal{S}}
\newcommand{\Q}{{\mathbb Q}}
\newcommand{\Z}{{\mathbb Z}}
%\newcommand{\N}{{\mathbb N}}
\newcommand{\C}{{\mathbb C}}
\newcommand{\R}{{\mathbb R}}
\newcommand{\F}{{\mathbb F}}
%\newcommand{\D}{{\mathbb D}}
\newcommand{\PP}{{\mathbb P}}
%\newcommand{\LL}{{\mathbb L}}
%\newcommand{\HH}{{G_{\LL}}}
\newcommand{\cA}{\mathcal{A}}
\newcommand{\cB}{\mathcal{B}}
\newcommand{\cG}{\mathcal{G}}
\newcommand{\cF}{\mathcal{F}}
\newcommand{\cH}{\mathcal{H}}
\newcommand{\cK}{\mathcal{K}}
\newcommand{\cI}{\mathcal{I}}
\newcommand{\cL}{\mathcal{L}}
\newcommand{\cM}{\mathcal{M}}
\newcommand{\cN}{\mathcal{N}}
\newcommand{\cP}{\mathcal{P}}
\newcommand{\cQ}{\mathcal{Q}}
\newcommand{\cR}{\mathcal{R}}
\newcommand{\cT}{\mathcal{T}}
\newcommand{\cJ}{{\mathcal J}}
\newcommand{\cC}{\mathcal C}
\newcommand{\cX}{\mathcal X}
\newcommand{\cV}{\mathcal V}
\newcommand{\cW}{\mathcal W}
\newcommand{\OO}{{\mathcal O}}
\newcommand{\ga}{\mathfrak{a}}
\newcommand{\gb}{{\mathfrak{b}}}
\newcommand{\gc}{{\mathfrak{c}}}
\newcommand{\cm}{\mathfrak{m}}
\newcommand{\ff}{\mathfrak{f}}
\newcommand{\gf}{\mathfrak{g}}
\newcommand{\gN}{\mathfrak{N}}
\newcommand{\pic}{{\rm Pic}^0(Y)}
\newcommand{\fp}{\mathfrak{p}}
\newcommand{\fq}{\mathfrak{q}}
\newcommand{\mP}{\mathfrak{P}}     
\newcommand{\xx}{\mathbf{x}}


%\renewcommand{\baselinestretch}{1.5}

\newtheorem{thm}{Theorem}
\newtheorem{fac}{Fact}
\newtheorem{conj}{Conjecture}
\newtheorem{prop}{Proposition}
\newtheorem{lem}{Lemma}
\newtheorem{alg}{Algorithm}
\newtheorem{cor}{Corollary}
\newtheorem{conc}{Conclusion}
\newtheorem{exercise}{Exercise}

\theoremstyle{definition}
\newtheorem{defn}{Definition}
\newtheorem{ex}{Example}

\theoremstyle{remark}

\newtheorem{ack}{Acknowledgement}


\begin{document}
\title{Chabauty and the Mordell--Weil Sieve\\
Episode 3} 
\author{Samir Siksek}
%\author{joint work with Nuno Freitas}
\institute{\Large\calligra University of Warwick}
\date{}

%gets rid of bottom navigation bars
\setbeamertemplate{footline}[page number]{}

%gets rid of navigation symbols
\setbeamertemplate{navigation symbols}{}

\frame{\titlepage}

\begin{comment}
\begin{frame}
Recall: given a curve $C$ over $\Q$, or over a number
field $k$, we want a complete description of $C(k)$.
For genus $\ge 1$, there is no known algorithm for giving this!
But there is a bag of tricks that can be used to show
that $C(k)$ is empty, or determine $C(k)$ if it is non-empty.
These include:
\begin{enumerate}
\item Quotients (lecture 1);
\item Descent (lecture 2);
\item Chabauty (lecture 3);
\item Mordell--Weil sieve (\textcolor{red}{today}).
\end{enumerate}

The purpose of these lectures is to get a feel for each of these
methods and see it applied to a particular example.
\end{frame}
\end{comment}

\begin{frame}
\frametitle{Recap--Jacobians}
Associated to a curve $C/k$ of genus $g\ge 1$ is a $g$-dimensional
abelian variety $J/k$.
\begin{enumerate}
\item[(i)] For $k$ a number field, $J(k)$
is a finitely generated abelian group (Mordell--Weil Theorem).
\item[(ii)] If $C(k) \ne \emptyset$ then $J(k) \cong \Pic^0(C/k)$
(the group of degree $0$ rational divisors on $C$ modulo principal divisors).
\item[(iii)] If $P_0 \in C(k)$, there is an embedding
\[
\iota \; : \; C \hookrightarrow J, \qquad P \mapsto [P-P_0]
\]
that is called the Abel--Jacobi map. We have $\iota(C(k)) \subseteq J(k)$.
\end{enumerate}
\end{frame}
\begin{frame}
\frametitle{Recap--Chabauty}
Let $C$ be a curve over $\Q_p$ ($p$ is a finite prime). Then there
is a pairing
\[
\langle \, , \, \rangle \; : \;
\Omega_{C} \times J_C(\Q_p) \rightarrow \Q_p,
\]
The pairing has the following properties:
\begin{enumerate}
\item it is $\Q_p$-linear on the left;
\item it is $\Z$-linear on the right;
\item the kernel on the right is $J(\Q_p)_{\mathrm{tors}}$ (the 
torsion subgroup of $J(\Q_p)$.
\end{enumerate}

\begin{lem}
Let $C$ be a curve over $\Q$ of genus $g$. Write 
$r$ for the rank of $J(\Q)$. Suppose $r\le g-1$. Let $p$
be a prime. Then there is some non-zero $\omega \in \Omega_{C/\Q_p}$
such that
\[
\text{$\langle \omega , D\rangle=0$ for all $D \in J(\Q)$}.
\]
\end{lem}
\begin{proof}
$\dim(\Omega_{C/\Q_p})=g$. Apply linear algebra.
\end{proof}
We call such $\omega$ an \textbf{annihilating differential}.
\end{frame}

\begin{frame}
If $P \in C(\Q_p)$ we define the \textbf{residue disk of $P$} by
\[
B_p(P) = \{ Q \in C(\Q_p) : Q \equiv P \pmod{p} \}.
\]
The number of residue disks is $\# C(\F_p)$.

Suppose $r < g-1$. Let $\omega$ be an annihilating differential,
and $P \in C(\Q)$. Chabauty's method gives a bound
$\Chab_p(P)$
for the number of points of rational points in the residue
disc of $P$:
\[
\# C(\Q) \cap B_p(P) \le \Chab_p(P).
\]
Let $\cK$ be the \textcolor{blue}{\textbf{known}} rational points. If $\# \cK \cap B_p(P) =
\Chab_p(P)$ then
\[
C(\Q) \cap B_p(P) =\cK \cap B_p(P). 
\]
I.e. we know all of the rational points in the residue disc of $P$.
\end{frame}
\begin{frame}
For Chabauty to succeed in finding $C(\Q)$, we need:
\begin{enumerate}
\item $r\le g-1$;
\item we need explicit generators for $J(\Q)$ (or some
subgroup of $J(\Q)$ of finite index);
\item we want some prime $p$ of good reduction so that
the known rational points surject onto $C(\F_p)$; 
\item  in each residue disc we want to find enough rational
points to match the Chabauty bound!
\end{enumerate}

Even if we have (1) and (2), we find in most examples that
(3) and (4) fail. 
\end{frame}
\begin{frame}
Let
\[
C \; : \; y^2=2x^6-3x^2-2x+1.
\]
A short search reveals the following four points:
\[
\cK=\{ (0,1),\; (0,-1), \; (-2,11),\; (-2,-11) \}.
\]
Then 
\[
J(\Q) = \Z \cdot [ (-2,-11) - (0,1)] \, .
\]
Annhilating differential for $p=3$ is
\[
\omega=
(66+O(3^5)) \frac{dx}{y} + \frac{x dx}{y} \, .\]
%and for $p=5$ is
%\[
%\omega=(294+O(5^5)) \frac{dx}{y}+ \frac{x dx}{y}.
%\]
%\end{frame}
%
%\begin{frame}
Applying Chabauty with $p=3$ we have

\begin{tabular}{|c|c|c|}
\hline
$P$ & $\Chab_3(P)$ & $\cK \cap B_3(P)$ \\
\hline
$(0,1)$ & $2$ & $\{(0,1)\}$ \\
$(0,-1)$ & $2$ & $\{(0,-1)\}$ \\
$(-2,11)$ & $1$ & $\{(-2,11)\}$\\
$(-2,-11)$ & $1$ & $ \{(-2,-11)\}$\\
\hline
\end{tabular}

\end{frame}
\begin{frame}
\begin{tabular}{|c|c|c|}
\hline
$P$ & $\Chab_3(P)$ & $\cK \cap B_3(P)$ \\
\hline
$(0,1)$ & $2$ & $\{(0,1)\}$ \\
$(0,-1)$ & $2$ & $\{(0,-1)\}$ \\
$(-2,11)$ & $1$ & $\{(-2,11)\}$\\
$(-2,-11)$ & $1$ & $ \{(-2,-11)\}$\\
\hline
\end{tabular}

\medskip

For $P=(-2,-11)$ and $(-2,11)$ there are no other rational 
points in the same residue disc. For $P=(0,1)$ and $P=(0,-1)$
we don't know.

\bigskip
Let
\[
B_9(P)=\{ Q \in C(\Q_3) : Q \equiv P \pmod{9} \}.
\]

\begin{tabular}{|c|c|c|}
\hline
$P$ & $\Chab_9(P)$ & $\cK \cap B_9(P)$ \\
\hline
$(0,1)$ & $1$ & $\{(0,1)\}$ \\
$(0,-1)$ & $1$ & $\{(0,-1)\}$ \\
\hline
\end{tabular}

\medskip

For $P=(0,1)$ and $(0,-1)$ there are no other rational 
points in the smaller residue disc $B_9(P)$. 
\end{frame}

\begin{frame}

Note
\[
C(\F_3)=\{ 
(\bar{0},\bar{1}) \, , \,
(\bar{0},\bar{2}) \, , \,
(\bar{1},\bar{1}) \, , \,
(\bar{1},\bar{2}) 
\}
\]

We know all the rational points in
\[
B_9(0,1) \cup B_9(0,-1) \cup B_3(-2,11) \cup B_3(-2,-11).
\]
This does not fill up $C(\Q_3)$.

To show that $\cK=\{(0,1),(0,-1),(-2,11),(-2,-11)\}$ is
all of the rational points, we need to show that every rational
point belongs to one of these four neighbourhoods. 
This is what  the {\bf Mordell--Weil} sieve will achieve.



\end{frame}

\begin{frame}
\frametitle{Mordell--Weil Sieve}
Let $P_0=(0,1)$. 
Let 
\[
\iota \; : \; C \hookrightarrow J, \qquad Q \mapsto [Q-P_0]
\]
be the associated Abel-Jacobi map.
Recall
\[
J(\Q)=\Z \cdot D, \qquad D=[(-2,-11)-(0,1)].
\]

Note that
\[
\iota(0,1)=0,\quad
\iota(0,-1)=-2D, \quad
\iota(-2,11)=-3D, \quad
\iota(-2,-11)=D.
\]

Suppose $Q \in C(\Q)$. Then $\iota(Q)=nD$ with $n \in \Z$.
We will use reduction mod $p$ for lots of primes $p$
to \lq predict\rq $n$.
\end{frame}

\begin{frame}
Suppose $Q \in C(\Q)$. Then $\iota(Q)=nD$ with $n \in \Z$.
We will use reduction mod $p$ for lots of primes $p$
to \lq predict\rq $n$.

\bigskip

Let $p$ be a prime of
of good reduction. Let 
\[
N=\text{order of $\bar{D} \in J(\F_p)$}. 
\]

Consider the commutative diagram
\[
\xymatrixcolsep{5pc}
\xymatrix{
C(\Q) \ar^{\iota}[r] \ar^{\mathrm{red}}[d]  & J(\Q) 
\ar[d]^{\mathrm{red}} & \Z \ar^{\eta}[l]    \ar[d]\\
C(\F_p) \ar^{\iota}[r] & J(\F_p)   & \Z/N\Z \ar^{\eta}[l] \\
}
\]
Here $\eta(m)=mD$.
By diagram chasing
\[
n \, \mathrm{mod} \, N \in \{ m \in \Z/N\Z : m \cdot \bar{D} \in \iota(C(\F_p)) \}.
\]
\end{frame}

\begin{comment}

\begin{frame}
Suppose $Q \in C(\Q)$. Then $\iota(Q)=nD$ with $n \in \Z$.
We will use reduction mod $p$ for lots of primes $p$
to \lq predict\rq $n$.

For every prime $p$ of good reduction, the Mordell--Weil sieve
gives an integer $N_p$ and a set $W_p$ such
that $n \, \mathrm{mod} \, N_p \in W_p$.

\bigskip

\begin{tabular}{|l|l|l|}
\hline
$p$ & $N_p$ & $W_p$\\
\hline
$3$ & $13$ & $\{ 0, 1, 10, 11 \}$ \\
$5$ & $21$ & $\{ 0, 1, 18, 19 \} $ \\
$7$ & $ 65$ & $\{  0, 1, 13, 19, 27, 36, 44, 50, 62, 63 \}$\\
$23$ & $16$ & $\{  0, 1, 7, 13, 14 \}$\\
$61$ & $208$ & $\{  0, 1, 24, 53, 153, 182, 205, 206\}$\\ 
\hline
\end{tabular}

\bigskip

Note that $16 \mid 208$.

\end{frame}


\begin{frame}
Suppose $Q \in C(\Q)$. Then $\iota(Q)=nD$ with $n \in \Z$.
We will use reduction mod $p$ for lots of primes $p$
to \lq predict\rq $n$.

For every prime $p$ of good reduction, the Mordell--Weil sieve
gives an integer $N_p$ and a set $W_p$ such
that $n \, \mathrm{mod} \, N_p \in W_p$.

\bigskip

\begin{tabular}{|l|l|l|}
\hline
$p$ & $N_p$ & $W_p$\\
\hline
$3$ & $13$ & $\{ 0, 1, 10, 11 \}$ \\
$5$ & $21$ & $\{ 0, 1, 18, 19 \} $ \\
$7$ & $ 65$ & $\{  0, 1, 13, 19, 27, 36, 44, 50, 62, 63 \}$\\
$23$ & $16$ & $\{  0, 1, 7, 13, 14 \}$\\
$61$ & $208$ & $\{  0, 1, \hcancel[red]{24}, 53, 153, 182, 205, 206\}$\\ 
\hline
\end{tabular}

\bigskip

Note that $16 \mid 208$.

\end{frame}

\begin{frame}
Suppose $Q \in C(\Q)$. Then $\iota(Q)=nD$ with $n \in \Z$.
We will use reduction mod $p$ for lots of primes $p$
to \lq predict\rq $n$.

For every prime $p$ of good reduction, the Mordell--Weil sieve
gives an integer $N_p$ and a set $W_p$ such
that $n \, \mathrm{mod} \, N_p \in W_p$.

\bigskip

\begin{tabular}{|l|l|l|}
\hline
$p$ & $N_p$ & $W_p$\\
\hline
$3$ & $13$ & $\{ 0, 1, 10, 11 \}$ \\
$5$ & $21$ & $\{ 0, 1, 18, 19 \} $ \\
$7$ & $ 65$ & $\{  0, 1, 13, 19, 27, 36, 44, 50, 62, 63 \}$\\
$23$ & $16$ & $\{  0, 1, 7, 13, 14 \}$\\
$61$ & $208$ & $\{  0, 1, \hcancel[red]{24, 53, 153, 182,} 205, 206\}$\\ 
\hline
\end{tabular}

\bigskip

Note that $13 \mid 65$.

\end{frame}

\begin{frame}
Suppose $Q \in C(\Q)$. Then $\iota(Q)=nD$ with $n \in \Z$.
We will use reduction mod $p$ for lots of primes $p$
to \lq predict\rq $n$.

For every prime $p$ of good reduction, the Mordell--Weil sieve
gives an integer $N_p$ and a set $W_p$ such
that $n \, \mathrm{mod} \, N_p \in W_p$.

\bigskip

\begin{tabular}{|l|l|l|}
\hline
$p$ & $N_p$ & $W_p$\\
\hline
$3$ & $13$ & $\{ 0, 1, 10, 11 \}$ \\
$5$ & $21$ & $\{ 0, 1, 18, 19 \} $ \\
$7$ & $ 65$ & $\{  0, 1, 13, \hcancel[red]{19}, 27, 36, \hcancel[red]{44}, 50, 62, 63 \}$\\
$61$ & $208$ & $\{  0, 1, \hcancel[red]{24, 53, 153, 182,} 205, 206\}$\\ 
\hline
\end{tabular}

\bigskip

Need more data.
\end{frame}
\end{comment}

\begin{frame}
Suppose $Q \in C(\Q)$. Then $\iota(Q)=nD$ with $n \in \Z$.
We will use reduction mod $p$ for lots of primes $p$
to \lq predict\rq $n$.

For every prime $p$ of good reduction, the Mordell--Weil sieve
gives an integer $N_p$ and a set $W_p$ such
that $n \, \mathrm{mod} \, N_p \in W_p$.

\bigskip

\begin{tabular}{|l|l|l|}
\hline
$p$ & $N_p$ & $W_p$\\
\hline
$3$ & $13$ & $\{ 0, 1, 10, 11 \}$ \\
$5$ & $21$ & $\{ 0, 1, 18, 19 \} $ \\
$7$ & $ 65$ & $\{  0, 1, 13, 19, 27, 36, 44, 50, 62, 63 \}$\\
$17$ & $ 39$ & $\{  0, 1, 36, 37 \}$\\ 
$19$ & $234$ & $\{0, 1, 42, 67, 72, 82, 100, 132, 150, 160, 165, 190, 231, 232\}$\\ 
$61$ & $208$ & $\{  0, 1, 24, 53, 153, 182, 205, 206\}$\\ 
\hline
\end{tabular}

\bigskip

Note that $39 \mid 234$.

\end{frame}


\begin{frame}
Suppose $Q \in C(\Q)$. Then $\iota(Q)=nD$ with $n \in \Z$.
We will use reduction mod $p$ for lots of primes $p$
to \lq predict\rq $n$.

For every prime $p$ of good reduction, the Mordell--Weil sieve
gives an integer $N_p$ and a set $W_p$ such
that $n \, \mathrm{mod} \, N_p \in W_p$.

\bigskip

\begin{tabular}{|l|l|l|}
\hline
$p$ & $N_p$ & $W_p$\\
\hline
$3$ & $13$ & $\{ 0, 1, 10, 11 \}$ \\
$5$ & $21$ & $\{ 0, 1, 18, 19 \} $ \\
$7$ & $ 65$ & $\{  0, 1, 13, 19, 27, 36, 44, 50, 62, 63 \}$\\
$17$ & $ 39$ & $\{  0, 1, 36, 37 \}$\\ 
$19$ & $234$ & $\{0, 1, \hcancel[red]{42, 67, 72, 82, 100, 132, 150, 160, 165, 190,} 231, 232\}$\\ 
$61$ & $208$ & $\{  0, 1, 24, 53, 153, 182, 205, 206\}$\\ 
\hline
\end{tabular}

\bigskip

Note that $39 \mid 234$.

\end{frame}
\begin{frame}
If $Q \in C(\Q)$ then $\iota(Q)=n D$
where $n \equiv 0,1,-3,-2 \pmod{234}$.

\bigskip

But
\[
\iota(0,1)=0,\quad
\iota(0,-1)=-2D, \quad
\iota(-2,11)=-3D, \quad
\iota(-2,-11)=D.
\]

Take $n \equiv -3 \pmod{234}$. So $n=-3+234m$. Then
\[
\begin{split}
[Q-P_0] &=\iota(Q)=nD\\
& =-3D+m(234\cdot D)\\
&=\iota(-2,11)+m(234 \cdot D)\\
& = [(-2,11)-P_0] + m (234 \cdot D).
\end{split}
\]
Hence $[Q-(-2,11)] = m(234 \cdot D)$.

\bigskip

\textbf{Conclusion:} if $Q \in C(\Q)$ then $\exists P
\in \cK$ such that
\[
[Q-P] \in \Z\cdot (234\cdot D) .
\]

\end{frame}

\begin{frame}
\frametitle{$p$-adic Filtration}
Let $p$ be a prime of good reduction. Let
\[
J^{m} (\Q_p)=\{ D \in J(\Q_p) : D \equiv 0 \pmod{p^m} \}.
\]
We have
\[
J(\Q_p) \supset J^{1}(\Q_p) \supset J^2(\Q_p) \supset J^3(\Q_p) \supset \cdots
\]
is a system of decreasing neighbourhoods of the origin.
Also
\[
J(\Q_p)/J^1(\Q_p) \cong J(\F_p), \qquad
J^{m}(\Q_p)/J^{m+1}(\Q_p) \cong (\Z/p\Z)^g \; \text{for $m \ge 1$}.
\]

\bigskip
For our example,
\[
\#J(\F_3)=13, \qquad 234=2 \cdot 3^2 \cdot 13.
\]
Hence $234 D \in J^{3}(\Q_3)$.
I.e. $234 D \equiv 0 \pmod{3^3}$.
\end{frame}

\begin{frame}
\frametitle{End of Example}
We have two important pieces of information:
\begin{enumerate}
\item If $Q \in C(\Q)$ then $\exists P
\in \cK$ such that
\[
[Q-P] \in\Z \cdot (234\cdot D) .
\]
\item $234 D \equiv 0 \pmod{3^3}$.
\end{enumerate}
Thus 
\[
Q \equiv P \pmod{3^3}, \qquad P \in \cK=
\{(0,1),(0,-1),(-2,11),(-2,-11)\}.
\]


So $Q$ belongs to
\begin{multline*}
B_{27}(0,1) \cup B_{27}(0,-1) \cup B_{27}(-2,11) \cup B_{27}(-2,-11)
\\
\subset
B_9(0,1) \cup B_9(0,-1) \cup B_3(-2,11) \cup B_3(-2,-11).
\end{multline*}

\pause

Thus (Chabauty and the Mordell--Weil sieve)
\[
C(\Q)=\{ (0,1), \; (0,-1), \; ( -2,11), \; (-2,-11) \}.
\]
\end{frame}


\begin{frame}
\frametitle{The Mordell--Weil Sieve}

Let $C/\Q$ be a curve, $J$ its Jacobian. Fix $P_0 \in J(\Q)$.  Let
\[
\iota \; : \; C \hookrightarrow J, \qquad P \mapsto [P-P_0]
\] 
be the Abel--Jacobi map. 
We assume that we know $J(\Q)$ (in other words,
we know a basis for $J(\Q)$). The {\bf Mordell--Weil Sieve}
is a strategy for producing a \lq small\rq\ finite set $W \subset
J(\Q)$, and a subgroup $L \subset J(\Q)$ of \lq huge\rq\ index
such that 
\[
\iota(C(\Q))=\bigcup_{D \in W} D+L
\pause
=:W+L.
\]
\end{frame}

\begin{frame}
\frametitle{Inductive Definition}
Let $C/\Q$ be a curve, $J$ its Jacobian. Fix $P_0 \in J(\Q)$.  Let
\[
\iota \; : \; C \hookrightarrow J, \qquad P \mapsto [P-P_0]
\] 
be the Abel--Jacobi map. 
We define inductively subgroups of finite index 
$L_i \subset J(\Q)$, and finite subsets
$W_i \subset J(\Q)$, such that
\[
L_0 \supseteq L_1 \supseteq L_2 \supseteq L_3 \supset \cdots
\]
and 
\[
\iota(C(\Q)) \subset W_i+L_i. 
\]
Start:
\[
L_0:=J(\Q), \qquad W_0:={0}.
\]
\end{frame}
\begin{frame}
Inductive Step: choose a prime $p$ of good reduction. Let
\[
L_{i+1}=\Ker\left(L_i \hookrightarrow J(\Q) \rightarrow J(\F_p) \right).
\]
Let
\[
W_{i+1}^\prime=W_i+ \left(L_i/L_{i+1} \right).
\]
Clearly $W_{i+1}^\prime+L_{i+1}=W_i+L_i$. So $\iota(C(\Q)) \subset W_{i+1}^\prime+L_{i+1}$.

\bigskip

Consider the commutative diagram
\[
\xymatrixcolsep{5pc}
\xymatrix{
C(\Q) \ar^{\iota}[r] \ar^{\mathrm{red}}[d]  & J(\Q) 
\ar[d]^{\mathrm{red}} \\
C(\F_p) \ar^{\iota}[r] & J(\F_p)   \\
}
\]
\end{frame}
\begin{frame}
Inductive Step: choose a prime $p$ of good reduction. Let
\[
L_{i+1}=\Ker\left(L_i \hookrightarrow J(\Q) \rightarrow J(\F_p) \right).
\]
Let
\[
W_{i+1}^\prime=W_i+ \left(L_i/L_{i+1} \right).
\]
Clearly $W_{i+1}^\prime+L_{i+1}=W_i+L_i$. So $\iota(C(\Q)) \subset W_{i+1}^\prime+L_{i+1}$.

\bigskip

Consider the commutative diagram
\[
\xymatrixcolsep{5pc}
\xymatrix{
C(\Q) \ar^{\iota}[r] \ar^{\mathrm{red}}[d]  & W_{i+1}^\prime+L_{i+1} 
\ar[d]^{\mathrm{red}} \\
C(\F_p) \ar^{\iota}[r] & J(\F_p)   \\
}
\]
\end{frame}
\begin{frame}
Inductive Step: choose a prime $p$ of good reduction. Let
\[
L_{i+1}=\Ker\left(L_i \hookrightarrow J(\Q) \rightarrow J(\F_p) \right).
\]
Let
\[
W_{i+1}^\prime=W_i+ \left(L_i/L_{i+1} \right).
\]
Clearly $W_{i+1}^\prime+L_{i+1}=W_i+L_i$. So $\iota(C(\Q)) \subset W_{i+1}^\prime+L_{i+1}$.

\bigskip

Consider the commutative diagram
\[
\xymatrixcolsep{5pc}
\xymatrix{
C(\Q) \ar^{\iota}[r] \ar^{\mathrm{red}}[d]  & W_{i+1}^\prime+L_{i+1} 
\ar[d]^{\mathrm{red}} \ar[rd] & \\
C(\F_p) \ar^{\iota}[r] & J(\F_p) & W_{i+1}^\prime \ar^{\red}[l]   \\
}
\]
\pause
Let
\[
W_{i+1}=\{w \in W_{i+1}^\prime \; : \; \red(w) \in \iota ( C(\F_p))\}.
\]

Then $\iota(C(\Q)) \subset W_{i+1}+L_{i+1}$.
\end{frame}
\begin{frame}
Choice of $p$:
\begin{enumerate}
\item $[L_i : L_{i+1}]$ is small;
\item $\# J(\F_p)$ is smooth.
\end{enumerate}

In practice, we usually find, \textbf{with a good strategy for choosing the $p$},
\[
W_i=\iota(\cK)\qquad \text{($\cK \subset C(\Q)$ are the known points)}
\]
for large, and the index $[J(\Q):L_i]$ is growing slowly.

\bigskip

The $L_i$ are decreasing neighbourhoods of the origin in the profinite topology.
When the Mordell--Weil sieve works, it tells us that every rational point on 
$C$ is close, in the profinite topology on $J(\Q)$, to one of the known ones.
\end{frame}

\begin{frame}
\begin{example}[Bugeaud, Mignotte, S., Stoll, Tengely]
\[
C \; : \; y^2-y=x^5-x,\qquad \iota:C \hookrightarrow J, \; P\mapsto [P-\infty].
\]
\[
J(\Q)=\Z \cdot D_1\oplus \Z \cdot D_2 \oplus \Z \cdot D_3,
\]
\[
D_1=[(0,1) -\infty], \qquad D_2=[(1,1)-\infty],
\qquad D_3=[(-1,1)-\infty]\, .
\]
The known rational points are
\begin{align*}
&\cK = \{\infty, \; (-1,0),\; (-1,1), \; (0,0),\; (0,1),\; (1,0),\; (1,1),\;(2,-5),\\
     & (2,6),\; (3,-15),\; (3,16),\; (30,-4929),\; (30,4930), \;
 (1/4, 15/32),\\
&  (1/4, 17/32),\;
 (-15/16, -185/1024),\;
(-15/16, 1209/1024) \}.
\end{align*}
Using $922$ prime $p <10^6$ it can be shown that
\[
\iota(C(\Q)) \subset \iota(\cK) +L
\]
where
\[
[J(\Q):L] \sim 3.32 \times 10^{3240}.
\]
\end{example}
\end{frame}
\begin{frame}
\begin{example}
\[
C \; : \; y^2-y=x^5-x,\qquad \iota:C \hookrightarrow J, \; P\mapsto [P-\infty].
\]
The known rational points are
\begin{align*}
&\cK = \{\infty, \; (-1,0),\; (-1,1), \; (0,0),\; (0,1),\; (1,0),\; (1,1),\;(2,-5),\\
     & (2,6),\; (3,-15),\; (3,16),\; (30,-4929),\; (30,4930), \;
 (1/4, 15/32),\\
&  (1/4, 17/32),\;
 (-15/16, -185/1024),\;
(-15/16, 1209/1024) \}.
\end{align*}
Using $922$ prime $p <10^6$ it can be shown that
\[
\iota(C(\Q)) \subset \iota(\cK) +L
\]
where
\[
[J(\Q):L] \sim 3.32 \times 10^{3240}.
\]
The shortest non-zero vector in $L$ has length $\sim 1.156 \times 10^{1080}$.
So if $P \in C(\Q) \backslash \cK$ then
\[
H(P) \ge \exp(10^{2160}).
\]
\end{example}
\end{frame}
\begin{frame}
\frametitle{Baker's Bounds}
Baker's theory tells us that if $P$ is an \textbf{integral point} then
\[
H(P) \le \exp(10^{565}).
\]
So we know all the integral points:
\begin{align*}
&C(\Z) = \{(-1,0),\; (-1,1), \; (0,0),\; (0,1),\; (1,0),\; (1,1),\;(2,-5),\\
     & (2,6),\; (3,-15),\; (3,16),\; (30,-4929),\; (30,4930) \}.
\end{align*}

How do you find the rational points on $C$?

\pause

\bigskip

\begin{center}
\Large{\textcolor{blue}{Thank you for your attention!}} \newline
\Huge{\textcolor{blue}{Huge thank you to the organizers!!}}
\end{center}

\end{frame}
\end{document}

